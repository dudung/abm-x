% 20190826
% It will be increased by 1 after \chapter
\setcounter{chapter}{0}
% Set this page number
\setcounter{page}{1}


\chapter{Pendahuluan}
Secara umum banyak sistem fisis yang dapat dimodelkan dengan menggunakan agen baik berupa butiran-butiran atau entitas lain yang saling berinteraksi baik secara permanen (rentang jauh) ataupun hanya sesekali (rentang dekat). Tulisan ini akan membahas hal tersebut yang dilengkapi dengan contoh implementasinya menggunakan pustaka \verb|abm-x|\footnote{url \url{https://github.com/dudung/abm-x} [20200605]} yang ditulis dalam bahasa pemrograman JavaScript (JS) dan lainnya.


%
\section{Berkas-berkas utama}
Terdapat beberapa berkas yang diperlukan untuk menjalankan pustaka \verb|abm-x|, yaitu berkas HTML dan JS yang dalam contoh ini bernama \verb|hello.html| dan \verb|hello.js|, serta versi terkompresi dari pustaka sebelumnya dengan nama berkas \verb|butiran.min.js| yang dapat diperoleh dari folder \verb|dist|.

\lstinputlisting[label={lst:hello.html}, caption={Berkas {\tt hello.html}.}, linerange={1}, firstnumber=1]{01/hello.html}

Kode \ref{lst:hello.html} memerlukan dua berkas yaitu versi terkompresi dari pustaka \verb|butiran.js| dan berkas \verb|hello.js|.

\lstinputlisting[label={lst:hello.js}, caption={Berkas {\tt hello.js}.}, linerange={1}, firstnumber=1]{01/hello.js}

Isi dari berkas \verb|hello.js| diberikan dalam Kode \ref{lst:hello.js}. Beris 18 akan menuliskan frasa "Hello world!" pada konsol perambat internet, e.g. Google Chrome, baris 19 mendefinisikan suatu besaran berjenis \verb|Vect3| dan baris 20 menuliskan isinya. Kelas \verb|Vect3| merupakan bagian dari pustaka \verb|butiran.js| yang bila tidak ada pesan kesalahan menunjukkan bahwa pustaka ini telah sukses disertakan.

\begin{figure}[H]
\centering
\includegraphics[width=10cm]{01/hello.png}
\caption{\label{fig:hello} Tampilan berkas {\tt hello.html} dalam peramban internet Google Chrome.}
\end{figure}

Untuk melihat konsol pada Google Chrome digunakan kombinasi tombol CTRL + SHIFT + J, yang berbeda-beda untuk setiap peramban internet.\footnote{"How to open the developer console", Airtable, url \url{https://support.airtable.com/hc/en-us/articles/232313848-How-to-open-the-developer-console} [20191017].} Dalam Gambar \ref{fig:hello} pada bagian kanan tercantum baris keberapa pada berkas \verb|hello.js| yang menghasilkan keluaran tersebut. Dengan menggunakan informasi ini pengguna dapat melacak hasil keluaran yang diperoleh.

Untuk berikutnya terkait dengan konsol bila tidak benar-benar diperlukan tampilan dalam peramban internet tidak akan ditayangkan, melainkan cukup hanya bagian teksnya tersebut

\begin{lstlisting}[numbers=none]
Hello, world!                                  hello.js:18
a = (0, 0, 0)                                  hello.js:20
\end{lstlisting}

yang ditampilkan.


%
\section{Catatan}
Rujukan, terutama yang yang bersumber dari internet, akan disertakan sebagai catatan kaki.